\chapter{BLOCK PREFETCHING}\label{block prefetching}

This chapter presents the idea of prefetching block data from server to hide the
network transfer time.

\section{Background}
To process a large array which cannot fit into memory of a single node, a typical
workflow in SIAL consists of requesting a block of array from server in a
\texttt{pardo} looping construct by each participating worker. After processing
it, the resulting block is sent back to server. This common pattern can be
summarized as single or multiple network bound operation surrounding one or more
compute bound operations.

%\hline
\begin{algorithm}  {Processing large array}
%\hline %
\singlespacing

\begin{algorithmic}[1]
%\caption{An example of typical workflow in SIAL}\label{alg:SIALWorkFLow}
\Loop\Comment{SIAL do/pardo loop}
\State $GET\ A[i, j]$\Comment{Request data from server asynchronously}
\State $GET\ B[j, k]$\Comment{Network bound}
\State $t\_result[j, k] \gets A[i, j] \times B[j, k]$\Comment{Compute bound}
\State $CALL\ compute\_fun(t\_result[j, k])$\Comment{Compute bound}
\State $PUT\ AB[i, k] \gets t\_result[i, k]$\Comment{Network bound}
\EndLoop
%\hline
\end{algorithmic}
\end{algorithm}

It is clear from the pseudocode that the computing resources are wasted while
waiting for the data to be ready. To improve the wait time of the compute
operation, non-blocking MPI call \texttt{MPI\_Irecv} was exploited to prefetch
the blocks from server over network.

\section{Implementation of Prefetching}

\subsection{pardo Loop Implementation}
While the \texttt{do} loop which iterates over the indices one by one in SIAL is
simple, there are multiple implementation of \texttt{pardo} loop, which differ
in the distribution of indices and thus distribution of blocks over workers:
\begin{itemize}
\item \texttt{SequentialPardoLoop}: it behaves similar to a simple do loop,
  except this loop can loop over multiple indices.
\item \texttt{StaticTaskAllocParallelPardoLoop}: the indices for this loop are
  determined statically by distributing the block over worker in a cyclic fashion.
\item \texttt{BalancedTaskAllocParallelPardoLoop}: to support symmetric arrays,
  SIAL has \texttt{where} construct in loops which prunes the iteration based on
  some programmer defined condition. Due to such pruning there is a non zero
  probability that all of the iterations are assigned to one particular worker.
  This loop evaluates the \texttt{where} clauses and distributes the valid iteration
  over workers in a balanced way.
\item \texttt{FragLoopManager} and its sub classes: SIAL supports large sparse
  arrays. To loop over them efficiently SIAL has various implementations of
  fragmented pardo loops. These loops have knowledge of internal structure of the
  sparse arrays and thus can skip over rows and columns having non useful values.
\end{itemize}
Due to so many varieties of implementation of \texttt{pardo} looping construct
and to support future implementations of indices generation schemes, it is
important to keep the mechanism of prefetching separate from indices generation.
For this a lazy prefetching mechanism was implemented which will probe for indices
as needed dynamically. A lazy implementation would also give freedom to vary number
of prefetched blocks at runtime.

\subsection{Lazy Indices Probing}
Each class implementing \texttt{pardo} have a function \texttt{update} which
calculates the values of set of indices and populates interpreter state. This
state is used by interpreter to calculate blocks using array and index values.

%\hline
\begin{algorithm}  {$update\_indices() \rightarrow bool$}
%\hline %
\singlespacing

\begin{algorithmic}[1]
%\caption{Processing large array in SIAL}\label{alg:euclid}
\Procedure{update\_indices}{}
\ForAll{$indices\ in\ loop$}
\State $old\_index\_val \gets
interpreter\_state.indices[index\_slot]$\Comment{get current index value}
\State $new\_val \gets old\_index\_val + 1$\Comment{Increment the index as needed}
\If{$new\_val \geq upper\_bound[index\_slot]$}
  \State $new\_val \gets lower\_seg[index\_slot]$
\EndIf
\State $interpreter\_state.indices[index\_slot] \gets new\_val$\Comment{update
  the interpreter state}
\EndFor
\If{$all\ indices\ reached\ upper\_bound$}
  \State \Return{false}
\Else
  \State \Return{true}
\EndIf
\EndProcedure
%\hline
\end{algorithmic}
\end{algorithm}

The mechanism to generate next indices and changing the state is
divided into two functions: \texttt{get\_next\_indices} and \texttt{update}.
Another function \texttt{peek\_indices} is defined to probe and form a list
of subsequent \texttt{n} indices.

\texttt{get\_next\_indices} is now changed to a function which produces set of
\textit{next} indices based on indices passed as parameter rather than getting
directly from interpreter state. This allows us to produce series of indices
independent of state of interpreter.

\begin{algorithm}  {get\_next\_indices([index]) $\rightarrow$ [index]}
\singlespacing

\begin{algorithmic}[1]
\Function{get\_next\_indices}{$current\_indices$}
\ForAll{$indx\_id\ in\ loop$}
\State $old\_index\_val \gets current\_indices[index\_id]$\Comment{get current index value}
\State $new\_val \gets old\_index\_val + 1$\Comment{Increment the index as needed}
\State $new\_indices \gets new\_val$\Comment{update the index into new set of indices}
\EndFor%
\State \Return{$new\_indices$}\Comment{return new set independent of interpreter state}
\EndFunction
%\hline
\end{algorithmic}
\end{algorithm}

Since the function \texttt{get\_next\_indices} is now independent of interpreter
state, it can be composed with other function calls other than \texttt{update}.
\texttt{peek\_indices(n)} maintains a list of indices using \texttt{get\_next\_indices}
and returns set of next indices if already computed or probes next set of indices
by calling \texttt{get\_next\_indices} on last set of calculated indices.

\begin{algorithm}  {peek_indices(IndexList::iterator it) $\rightarrow$ [index]}
\singlespacing

\begin{algorithmic}[1]
  \Function{peek\_indices}{$Indexlist::iterator it$}
  \If{$IndexList.empty()$}
    \State \Return $[\ ]$
  \Else
    \State $peekedIndices \gets *it$
    \If{$next(it) == IndexList.end()$}
      \State $get\_next\_indices(*it)$
    \EndIf
    \State \Return $peekedIndices$
  \EndIf
  \EndFunction
%\hline
\end{algorithmic}
\end{algorithm}

\texttt{peek\_indices} always appends to calculated indices or returns set of indices
if already calculated. The updated function \texttt{update\_indices} pops first
set of indices from the calculated indices list and update the interpreter
global state so that other components can find it. The function
\texttt{get\_next\_indices} requires set current indices to calculate set of next
indices, that is why there is a need to keep at least one element in the list of
calculated indices. \texttt{peek\_indices} makes sure there is always at least one
element in the list, hence \texttt{update\_indices} calls \texttt{peek\_indices}
to delegate this responsibility instead of directly getting from the list.

\begin{algorithm}  {update\_indices() $\rightarrow$ bool}
\singlespacing

\begin{algorithmic}[1]
\Function{update\_indices}{}
\State $current\_indices \gets peek\_indices(computed\_indices, 1)$
\If{$length(current\_indices) > 0$}\Comment{Base case for recursive call}
  \State $pop(computed\_indices)$
  \ForAll{$index\_slot\ in\ current\_indices $}
    \State $interpreter\_state.indices[index\_slot] \gets current\_indices[index\_slot]$
  \EndFor
  \State \Return $true$
\Else
  \State \Return $false$
\EndIf
\EndFunction
%\hline
\end{algorithmic}
\end{algorithm}