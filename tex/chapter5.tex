\chapter{EXPLOITING GPU}\label{exploitinggpu}
\section{Background}
\texttt{do} and \texttt{pardo} looping constructs are one of the most important
constructs in SIAL since they allow SIAL programmers to express operations on
blocks. Using \texttt{pardo} construct the calculations can be done in parallel.
SIAL runtime is responsible for the distribution of work over all the worker nodes.
In this chapter we present the problems faced in using GPU to execute the looping
constructs and how they were solved.

\subsection{Attempts in ACESIII}
There have been attempts\cite{jindal2016gpusial} made in previous version of ACES
to use GPU to speed up computation. In this work the SIAL programmer had to deal
with a lot of low level GPU memory operations such as allocating memory on GPU,
copying blocks to and from GPU and deallocating memory on GPU. Since not all
calculations are suitable to be executed on GPU, the SIAL programmer had to mark
regions of SIAL code suitable to be executed on GPU.

\begin{lstlisting}[caption={Code Fragment from ACESIII for CCSD calculation},
  label={lst:ACESIII_gpucode}]
#start of GPU region
gpu_begin
#allocate and initialize blocks on GPU
gpu_put aoint(lambda,mu,sigma,nu) #allocate and copy data from CPU
DO i1
DO j1
    gpu_put LT2AOab1(mu,i1,nu,j1)
    gpu_put LT2AOab2(nu,j1,mu,i1)
    gpu_put LTAOab(lambda,i1,sigma,j1)
ENDDO j1
ENDDO i1
gpu begin
DO i
DO j
    #perform computations on GPU
    Yab(mu,i,nu,j) = 0.0
    Y1ab(nu,j,mu,i) = 0.0
    gpu_allocate Yab(mu,i,nu,j) # allocate temp blocks on GPU
    gpu_allocate Y1ab(nu , j ,mu, i )
    #contraction Y1ab(nu,j,mu,i) = Yab(mu,i,nu,j) #permutation
    Yab(mu,i,nu,j) = aoint(lambda,mu,sigma,nu)*LTAOab(lambda,i,sigma,j)
    LT2AOab1(mu,i,nu,j) += Yab(mu,i,nu,j) #element−wise sums
    LT2AOab2(nu,j,mu,i) += Y1ab(nu,j,mu,i) #element−wise sums
    gpu_free Yab(mu,i,nu,j) #free temp blocks on GPU
    gpu_free Y1ab(nu,j,mu,i)
ENDDO j
ENDDO i
#copy results to CPU , free blocks on GPU
DO i1
DO j1
    gpu_get LT2AOab1(mu,i1,nu,j1)
    gpu_get LT2AOab2(nu,j1,mu,i1)
    gpu_free LT2AOab1(mu,i1,nu,j1)
    gpu_free LT2AOab2(nu,j1,mu,i1)
    gpu_free LTAOab(lambda,i1,sigma,j1)
ENDDO j1
ENDDO i1
gpu_free aoint(lambda,mu,sigma,nu)
gpu_end
#end of GPU region
\end{lstlisting}

A code fragment from ACESIII for CCSD calculation is presented in
\ref{lst:ACESIII_gpucode}. In line 2 and 39 the region is marked to be executed
on GPU and blocks of lines 5-11 and 29-37 deal with managing memory to and from
GPU and main memory. The actual calculations is done by lines 13-27.

\section{Runtime Memory Management}
To manage the block memory and to automate the memory transfer between GPU and CPU
meta data about state of memory was stored in interpreter. The \texttt{Block} class
which represents the super \textit{block} in interpreter was modified to now store
this meta data and pointer to memory location in GPU and main memory in form of
object of another class \texttt{Device\_Info}.

Due to this added layer, supporting multiple compute devices is now possible.
memory is also added to \texttt{Device\_Info} class. The meta data includes whether
the block data was \textit{dirty}, \textit{valid} and a \textbf{version number}
which helps in keeping block data allocated on GPU and main memory (CPU)
synchronized.

\begin{lstlisting}[caption={\texttt{Device\_Info} Class structure},
  language=C++,
  label={lst:device_info_structure}]
class DeviceInfo {
  double* data_ptr_;
  unsigned int data_version_;
  bool onDevice; // is block data on current device
  bool isDirty;  // is block data modified/dirty
  bool isAsync;  // are there any pending operations on device
}
\end{lstlisting}

The code shown in~\ref{lst:device_info_structure} is one of the way of implementing
the class \texttt{Device\_Info} in C++ language.

Using the meta data and specifically \texttt{data\_version\_}, the runtime can keep
track of changes in data on different devices and find the device with latest
data by comparing version numbers:

\begin{algorithm}  {Block::get\_latest\_data() $\rightarrow$ Device\_Info}
  \singlespacing

  \begin{algorithmic}[1]
    \Function{Block::get\_latest\_data}{}
    \State $highest\_version\_found \gets 0$
    \State $highest\_version\_device \gets null$
    \ForAll{$device\_info\ in\ this->devices$}
    \If{$device\_info.data\_version > highest\_version\_found$}
    \State {$highest\_version\_found \gets info.data\_version$}
    \State {$highest\_version\_device \gets device\_info$}
    \EndIf
    \EndFor
    \State \Return $highest\_version\_device$
    \EndFunction
    % \hline
  \end{algorithmic}
\end{algorithm}

After determining the device with latest version of data, the runtime can update
other devices if needed. The only interfacing function exposed outside of
\texttt{Block} class is \texttt{get\_data(deviceid)} which returns pointer to memory
location on device identified by \texttt{deviceid}. The logic to always return
latest version of data can be embedded into \texttt{get\_data(device)}:

\begin{algorithm}  {Block::get\_data(deviceid) $\rightarrow$ double*}
  \singlespacing

  \begin{algorithmic}[1]
    \Function{Block::get\_data}{}
    \State $latest\_device \gets this->get\_latest\_data()$
    \If{$latest\_device.data\_version > this->devices[deviceid]$}
    \State {$memcpy(latest\_device.data,\ this->devices[deviceid].data)$}
    \EndIf
    \State \Return $this->devices[deviceid].data$
    \EndFunction
    % \hline
  \end{algorithmic}
\end{algorithm}

\section{Reducing Block Transfers}
The SIAL compiler has information about block read/write access. Thus in every
operation the intent


\section{Memory Pinning}
This section presents technique of memory pinning which when used results in
faster copying of data between CPU main memory and GPU memory.

\subsection{Memcpy without Pinning}
When memory is allocated using usual constructs such as \textbf{new} or
\textbf{malloc}, the data allocated is pageable by default. The GPU cannot
directly access data allocated in pageable host main memory. So when data
movement is requested using CUDA functions such as \texttt{cuda\_memcpy}, the
CUDA driver has to first allocate a temporary page-locked, or ``pinned'', host
memory, copy the host data to pinned memory and then transfer the data from
pinned memory to device memory.

- Async Memcpy: eager push

\section{CUDA aware MPI}
- MPI get async