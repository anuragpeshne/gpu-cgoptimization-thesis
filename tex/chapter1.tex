\chapter{INTRODUCTION TO THESIS}\label{intro}
The Super Instruction Architecture (SIA) is a parallel programming environment
originally designed for problems in computational chemistry involving complicated
expressions defined in terms of tensors. Tensors are represented by
multidimensional arrays which are typically very large. The SIA consists of a
domain specific programming language, Super Instruction Assembly Language
(SIAL), and its runtime system, Super Instruction Processor. An important
feature of SIAL is that algorithms are expressed in terms of blocks or
multidimensional arrays rather than individual floating point numbers. This thesis
presents two ideas to enhance the looping constructs in SIA.

\section{Introduction to Issues with Working with GPUs}
In ACESIII, the previous version of ACES, programmer had to explicitly deal with managing
the memory transfer between GPU and CPU and marking regions well suited for execution
on GPU. In ACES4, managing memory transfer is done automatically by the runtime.
This is implemented by tracking block changes using version numbers.

Still there is need for SIAL programmer to decide which portion of the SIAL code
is well suited for execution on GPU. Transferring data between GPU and CPU is
expensive and can dominate the total time spend in computation. Hence this is not
a trivial decision to make since the programmer has to minimize time spend in
total memory transfers. This in turn depends on various factors such as
size of the block and number of instructions in the code block supported on GPU. Judging
based on size of block is itself difficult because it depends on the type of GPU available
at runtime. Larger blocks need more time to transfer between GPU and CPU, but at
same time the difference in time taken to operate on blocks by GPU and CPU grows
exponentially with the size of the block. And lastly, same programs are used to
calculate different results by supplying different data files. A portion of code
which executed faster on GPU for certain block size may in fact execute slower
than on CPU for smaller block size if the gain in time by executing on GPU cannot
compensate for the transfer time.

\section{Introduction to Issue of Data Transfer from Server}
In SIAL, looping constructs are the only way to work with individual blocks.
Typical work~flow of SIAL includes requesting a block of larger array from server,
processing the block, compute resulting block and send it back to the server. Though
the operation of requesting block from server is non blocking, subsequent operations
in the loop block until MPI transfer is completed. To amortize the cost of network
transfer, prefetching has been implemented which does the transfer concurrently
with the block processing. By prefetching the anticipated blocks that would be
needed in next iteration of loop, the wait~time for block can be reduced and
in some cases completely eliminated.

\section{Organization of the Thesis}
Chapter 2 presents related literature regarding efforts made to exploit GPU in
SIA as well as other programming models and the technique of prefetching to hide
latency in accessing data. Then an introduction to background of this thesis,
including architecture and implementation of SIA and ACES4, is presented in chapter~3.
Chapter 4 describes the problem and solution of optimizing use of GPU to execute SIAL
code; the design and implementation of eager-pushing of blocks and dynamic
determination of suitable code block for execution on GPU is presented in
chapter 5. Chapter 6 states the results of the benchmarks and experiments. And
finally, chapter 7 presents conclusion of this thesis and future works.