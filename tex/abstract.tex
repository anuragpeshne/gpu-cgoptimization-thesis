% Write in only the text of your abstract, all the extra heading jargon is automatically taken care of
\begin{abstract}
  The Super Instruction Architecture (SIA) is a parallel programming environment
  engineered to solve computation involving very large, possibly sparse, multidimentional arrays.
  SIA consists of a domain specific programming language, SIAL, and its runtime.
  A novel feature of the programming language is that SIAL programmers express
  algorithms in terms of operations on blocks rather than individual floating point
  numbers. Two enhancements, improvement in the GPU support and reduction of the network
  wait time using data prefetching, are presented in this work.\\

  In the previous version of SIAL, the domain programmer was required to manually transfer blocks between main memory
  and GPU memory as well as to mark parts of SIAL code to be executed on GPU. Copying data
  between GPU memory and main memory is expensive and can have a high impact on overall
  performance of GPU execution. In this work, automatic memory management is developed
  and several techniques, that reduce and in some cases eliminate cost of memory transfer
  between GPU and main memory, are implemented.\\

  In SIAL, there is a common code pattern, to request a block of data from a server
  followed by computation on that block and finally sending it back to a server,
  in a loop. This pattern makes inefficient use of network
  and computational resources, since one of them remains underutilized while the other is being used.
  In order to tackle this issue, deterministic request for blocks has been exploited
  to implement prefetching to optimally utilize the network resources and the
  computational resources.\\

  Various experiments have been conducted to evaluate the effect of change of various
  parameters on GPU utilization, prefetching, \texttt{wait\_time} and overall
  computation time.
\end{abstract}
