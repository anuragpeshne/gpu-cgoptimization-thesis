% Write in only the text of your abstract, all the extra heading jargon is automatically taken care of
\begin{abstract}
  The Super Instruction Architecture (SIA) is a parallel programming environment
  engineered to solve computation involving very large, possibly sparse, multidimentional arrays.
  SIA consists of a domain specific programming language, SIAL, and its runtime. ACES4
  is a computational chemistry package built using the SIA.
  A novel feature of the programming language is that SIAL provides intrinsic support
  for partitioning arrays into blocks and the domain programmers express
  algorithms in terms of operations on blocks rather than individual floating point
  numbers. Two enhancements, improvement in the GPU support and hiding of the network
  latency using data prefetching, are presented in this work.\\

  In ACESIII, the previous realization of the SIA, the SIAL programmer was required to manually transfer blocks between host memory
  and GPU memory as well as to explicitly mark parts of SIAL code to be executed on the GPU. Copying data
  between GPU memory and host memory is expensive and can have a high impact on overall
  performance of GPU execution. In this work, automatic memory management is developed
  and several techniques that reduce and in some cases eliminate cost of memory transfer
  between GPU and host memory are implemented.\\

  A common code pattern in SIAL is to request a block of data from a server
  followed by computation on that block, followed by sending it back to a server,
  in a loop. This pattern makes inefficient use of network
  and computational resources, since one of them remains underutilized while the other is being used.
  In order to tackle this issue, the runtime has been extended to automatically prefetch
  data blocks. Empirical results indicate the optimal number of blocks to prefetch.
\end{abstract}
