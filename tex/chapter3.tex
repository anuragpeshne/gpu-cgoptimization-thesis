\chapter{RELATED WORK}\label{lit}
In this chapter, prevous work done by others on General Purpose computing on Graphics Processing
Units (GPGPU) as well as using GPUs specifically for evaluating scientific calculations
and using prefetching to hide latency in accessing data, is discussed.

\section{GPGPU}
This section discusses previous developments in porting existing
code to GPUs using directives, using CUDA directly to speed up scientific calculations and
techniques to efficiently transfer data to GPU memory.

\subsection{Directive Based GPU programming}
There have been several developments made in the area of directive-based programming models. These
developments include OpenACC~\cite{openacc}, OpenMP for
accelerator~\cite{openmpforaccelerators}, and less updated developments including
OpenMPC~\cite{openmpc} and hiCuda~\cite{hicuda}. The approach taken by these models
is to make an implementation GPU-ready by placing directives which make
execution of loops parallel.

Directives used in ACESIII, were to manage
data transfer between GPU and host memory and to mark region of code suitable for
execution on GPUs. These directives are very different from the directives mentioned
in the above models, which hide the complexities of working with GPUs and allows
developers to work at higher abstraction than CUDA.
However, these models can be used to make an existing CPU implementation
of a super instruction, GPU-ready, by inserting suitable directives.

\subsection{GPU in Computational Chemistry}
Various implementations of coupled cluster methods, an important class of problems
in computational chemistry,
on GPUs have been reported~\cite{bhaskar2013, deprince2011, maw2011}. In these
implementations, a specific algorithm was selected and then implemented on a
GPU in a highly optimized form. These implementations have hardware specific
optimizations, and are not generic enough to remain effective as new hardware
development takes place.

One of the implementations of tensor contraction operators on a GPU by
Ma~et~al.~\cite{Ma2013} generates CUDA code to directly implement the contractions
by optimizing for the particular order of indices in the contractions. This, as
compared to using permutations and then two-dimensional matrix-matrix multiplication
as implemented in SIA, should achieve better performance for a contraction. These
specialized CUDA contraction implementations were added to NWChem, a computational
chemistry package, and considerable speedups were reported for CCSD(T) calculation
on CPU/GPU hybrid systems. Such optimized operators could be incorporated into SIA
as built-in super instruction for contraction.

\subsection{GPU Programming in Other High-Level Programming Languages}
Several attempts have been made to support GPUs in general purpose high-level programming
languages. A few examples include PyCUDA~\cite{pycuda2011},
jCUDA~\cite{jcuda2009} and Chapel~\cite{chapelgpu}. These toolkits provide an interface
to GPUs from high-level programming languages such as Python and Java, but the
programmer still needs to deal with low-level GPU bookkeeping such as memory
management and defining kernels. As described in section~\ref{relatedworkacesiiigpu},
this is similar to ACESIII, where the programmer has to manually transfer memory
between GPU and CPU. However, SIAL is a special purpose language and SIA runtime
has information about block sizes and block operations. GPU implementation for
some of the common block operations can be defined into the language itself. With
such implementations and availability of the information about block, the runtime can jump between
GPU and CPU implementation, transparently to the domain programmer, along with taking
care of memory management.

\subsection{GPUs in Previous Versions of ACES}\label{relatedworkacesiiigpu}
Work has been done to exploit GPUs in previous version of ACES, ACESIII\cite{Jindal2016}.
This implementation
required the programmer to mark a region of code to be executed on a GPU using
directives \texttt{gpu\_on} and \texttt{gpu\_off} and manually manage memory
transfer to and from the GPU. This puts a burden of managing memory and recognizing a
suitable suitable region of code for execution on GPUs on the domain programmer.
If a region of SIAL code that is not suitable for execution on GPUs is marked, it can result
in suboptimal use of resources. Executing operations using small blocks on GPUs can
be slower than executing on the CPU due to time spent in memory transfer and
insignificant gain in speed by execution. It can be a difficult decision for the
SIAL programmer to determine if a block is large enough so that benefit of executing
it on GPUs becomes significant.

\subsection{Optimizing GPU Memory Transfer}
\subsubsection{Host Memory Transfer}
GPUs cannot work directly with pageable host memory. One of the key condition for
effectively exploiting GPUs is to optimally manage data transfer between host memory
and GPU memory. This is true for any computation running on GPUs and not just High
Performance Computing computations. The work done by Fatica~\cite{Fatica:2009:ALC:1513895.1513901},
in accelerating LINPACK describes various ways for optimizing memory transfers. Their
implementation intercepts calls to \texttt{DGEMM} and \texttt{DTRSM} and executes them simultaneously
on both GPUs and CPU cores. To achieve faster memory transfer, the implementation uses fast
\textit{PCI-E} transfers by page locking or \textit{pinning} the host memory. Using
PCI-E transfer, they observed a speed improvement of 5 GB/s from 2 GB/s.
Kaldewey~et~al~\cite{Kaldewey:2012:GJP:2236584.2236592} describes similar technique
to optimize memory transfer using common address space for CPUs and GPUs, known as
Unified Virtual Addressing (UVA). This helps memory transfers in two ways: it
allows GPUs to work with memory greater than available physical GPU memory and also relieves
the programmer from the burden of managing two memory spaces. UVA requires that the
host memory is page locked. Using UVA, sustained rate of 6.2 GB/s has been reported in their work.
Host memory page locking is a common technique in both of the works, which enables GPU DMA controller
to access host memory at PCI-E speeds. This technique can be used to speed up the
SIA block transfers. This can be incorporated into an automatic memory synchronization module
such that it becomes transparent to the domain programmer.

\subsubsection{Network Transfer}
GPUs are often deployed in a cluster to accelerate computationally intensive
general-purpose tasks. Still, for communication with other nodes, the GPUs need assistance from the CPUs.
This requires intermediate copies from the GPU memory to the host memory. To have
a way around this inefficiency, there has been work done to make GPUDirect Remote
Direct Memory Access (RDMA) available for InfiniBand clusters. Shainer~et~al~\cite{Shainer2011} introduced
GPUDirect that enables GPUs to transfer data via InfiniBand without involvement
of the CPU or need of any buffer copies. Using GPUDirect RDMA, third-party devices such as
network adapters can access the GPU memory buffer directly, over the PCI-E bus.
Potluri~et~al~\cite{Potluri:2013:EIM:2570457.2571010} evaluated the GPUDirect RDMA for InfiniBand. They reported
improvement in latency of MPI Send/MPI Recv by 69\% and 32\% for message size of
4Byte and 128KByte and improvement in uni-directional bandwidth achieved using
4KByte and 64KByte messages by 2x and 35\%, respectively.
Although, GPUs are not allocated to the SIA servers, GPUDirect feature can still
be exploited to bypass the worker CPUs and to push the GPU memory buffer directly to the network
adapter. This can also be used to completely avoid host memory by collecting the
SIA block in GPU memory, operating on it using GPUs and sending it back, bypassing
the CPU. This can help in reducing host and GPU memory transfers.

\section{Prefetching}
In this section we look at work done using the general idea of prefetching such as
nonblocking fetch operations, techniques used to determine optimal prefetching
parameters such as the number of blocks to prefetch and the prefetch cache size.

Prefetching is an old technique~\cite{anacker68, Smith1982, Vanderwiel2000}
used to move data up in the memory hierarchy before it is actually needed by the processor.
The main aim behind prefetching is to hide the data access latency due to the difference in speed to access
data. A variety of approaches, including hardware prefetching, and caching, compiler
techniques, pre-execution and runtime execution, have been implemented in field of
high performance computing.

Hardware prefetching techniques, which includes transferring separate
cache blocks~\cite{Smith1978}, were implemented before software techniques became
available. Porterfield~\cite{Porterfield1989} introduced the idea of software prefetching
using special instruction to preload values into the cache without blocking the
computation. Using this instruction, the compiler can inform the cache over 100
cycles before a load is required. Asynchronous MPI message can be used to request
the server for the next block without waiting for the data transfer to complete. The runtime can
request for blocks several iteration before the blocks are required.

Prefetching is also exploited in the area of disk IO apart from feeding processor cache
from memory. Patterson et al~\cite{Patterson1994} implemented informed prefetching
mechanism for IO intensive application to exploit highly parallel IO systems. The
mechanism depends on disclosure of future accesses dynamically. In the SIA, the runtime
has information about how the array is going to be accessed by workers. Using this
information it can predict accurately which blocks should be prefetched. It can
also predict the order of blocks needed.

%Dahlgren and Stenström~\cite{Dahlgren1993} developed an algorithm to determine the
%number of memory blocks to prefetch. They proposed
%\textit{adaptive sequential prefetching} policy, which allows the number of blocks
%to prefetch, K, to vary during runtime of the program to reflect spatial locality
%shown by the program. K is varied by calculating \textit{prefetch efficiency} metric
%which is defined as the ratio of useful prefetched blocks to the total number of
%prefetched blocks. If the prefetch efficiency exceeds an upper threshold then K is
%incremented and decremented if it drops below a lower threshold. The value of K
%can reach 0, effectively disabling prefetching. Although, the SIA runtime can
%predict accurately the useful blocks, varying the number of blocks to prefetch, dynamically,
%can help control the network traffic. Due to several prefetching requests
%the network might get congested, resulting in delayed transfer of blocks. Workers
%can vary the number of blocks to prefetch by sensing delay in transfer of blocks.

There has been work done by Bhatia et al~\cite{Bhatia2010} to determine the size of
the cache, to maximize hit rate in a sequential prefetching scheme. In this work, an
online algorithm is devised which saves the blocks evicted from prefetch cache
into another cache, \textit{evicted cache}. If the incoming request is satisfied
by the evicted cache, according to the algorithm the size of prefetch cache is too
small and it increases it. If the request hits prefetch cache or there is a miss
on both of caches then the algorithm leaves the size unchanged. In order to determine if
the cache size is too large, the eviction cache is observed and if it does not
receives any hits then the size of the cache is decremented. In SIA, scope rules
indicate when blocks can be evicted.

In SIA, blocks which are needed to be prefetched can be predicted with more accuracy
than memory blocks. Further more, the exact sequence in which blocks are needed is also known.
This makes prefetching in SIA free of few of the problems faced in general memory
block prefetching discussed in above section. However, a large of number of blocks
should not be prefetched at once, since it might cause network congestion.
