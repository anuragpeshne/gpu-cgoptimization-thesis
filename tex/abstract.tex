% Write in only the text of your abstract, all the extra heading jargon is automatically taken care of
\begin{abstract}
  The Super Instruction Architecture (SIA) is a parallel programming environment
  engineered to work on large blocks of floating point numbers. Looping constructs
  \texttt{do} and \texttt{pardo} are one of the most important constructs in Super
  Instruction Assembly Language (SIAL) since they allow the programmer to work
  with individual block. To improve the runtime performance of the looping constructs
  two techniques are used: improving GPU utilization and using prefetching to
  hide network latency.\\

  In previous versions, the domain programmer was needed to manually copy main memory
  and GPU memory and to mark parts of SIAL code to be executed on GPU. Copying data
  between GPU memory and main memory is costly and have high impact on overall
  performance of GPU execution. For this reason choosing correct block of code is
  very crucial to overall performance. Automatic memory management is provided
  and various techniques are implemented to reduce and in some cases eliminate
  GPU memory and main memory transfer cost.\\

  There is a common code pattern in SIAL to request block of data followed by
  computation on that block. Deterministic request of blocks have been exploited
  to implement prefetching to optimally utilize network resources during the time
  when compute resources have blocked the control flow.\\

  Various experiments have been conducted to evaluate effect of change of various
  parameters on GPU utilization, prefetching, \texttt{wait\_time} and overall
  computation time.
\end{abstract}
