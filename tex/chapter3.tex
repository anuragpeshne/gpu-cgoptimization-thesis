\chapter{SUPER INSTRUCTION ARCHITECTURE} \label{materials}

This chapter introduces the Super Instruction Architecture (SIA), including the
design of worker and server, SIAL program, parallel looping constructs and
design of GPU implementation.

\section{Overview of the ACES}
ACES4 is a parallel application for chemical computation. It runs on a variety
of architectures, but is especially geared towards enabling calculations on
leadership class supercomputers. It can be deployed and scaled on multiple nodes
in a high performance cluster using Message Passing Interface (MPI) for inter
process communication. MPI processes or ranks in ACES4 are divided into servers
and workers. While workers do the actual computation, servers facilitate data to
workers.

The chemical computation done using ACES4 uses data of high dimension. A typical
calculation in this domain takes as input the geometry of a molecular system and
a choice of single particle orbitals as basis to expand the many-electron
quantum-mechanical wave function. The complex algorithms which produce
properties of the molecular system can easily require arrays of double precision
floating point of size several hundred Gigabytes. Of these arrays, at least
three need rapid access and are usually stored in RAM, the rest that are used
less frequently can be stored on disk.

\section{SIA}
The Super Instruction Architecture is a parallel programming environment
designed for computation dominated by tensor algebra with very large dense
arrays. Since the arrays are too fit to into memory of a single processor, SIA
introduces notion of blocks and super instructions. Each dimension of a array is
broken into \textit{segments} which represent blocks of floating point numbers.
The blocks are operated upon by super instructions which take a variable number
of blocks and output a block as result.

\subsection{Super Instruction Assembly Language}
The SIA consists of a domain specific programming language, Super Instruction
Assembly Language (SIAL) and its runtime system, the Super Instruction Processor
(SIP). Computational chemist express their algorithms in SIAL, a simple parallel
programming language with support for blocks, parallel looping construct and
operator which operate directly on blocks.

Thus SIAL helps chemists to focus on domain problem rather than worrying about
execution details and lets high performance computing specialist fine tune the
performance of ACES4. The SIAL programs compile to SIA byte code which is
interpreted dynamically. SIA interpreter manages runtime issues such as parallel
scheduling on processors, data communication, managing memory and even target of
execution of the byte code.

\section{Augue sapien mattis leo}
Nec accumsan turpis quam at neque. Ut pellentesque velit sed placerat cursus. Integer congue urna non massa dictum, a pellentesque arcu accumsan. Nulla posuere, elit accumsan eleifend elementum, ipsum massa tristique metus, in ornare neque nisl sed odio. Nullam eget elementum nisi. Duis a consectetur erat, sit amet malesuada sapien. Aliquam nec sapien et leo sagittis porttitor at ut lacus. Vivamus vulputate elit vitae libero condimentum dictum. Nulla facilisi. Quisque non nibh et massa ullamcorper iaculis.

