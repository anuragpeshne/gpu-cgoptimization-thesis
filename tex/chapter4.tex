\chapter{BLOCK PREFETCHING}\label{block prefetching}

This chapter presents the idea of prefetching data block from a server to hide the
network transfer time.

\section{Background}\label{prefetchingbackground}
To process a large array, which cannot fit into the memory of a single node, a typical
workflow in SIAL consists of requesting a block of an array from a server in a
\texttt{pardo} looping construct by each participating worker. After processing
it, the resulting block is sent back to a server. This common pattern can be
summarized as single or multiple network bound operation surrounding one or more
compute bound operations.

\begin{lstlisting}[caption={Code fragment from ACESIII for CCSD calculation},
  label={alg:SIALWorkFLow}]
PARDO i1, i, a, i2  # parallel looping construct

  # sends a request for block of array VSpipi to the server
  # Asynchronous operation, the runtime doesn't wait for the transfer to be completed
  GET VSpipi[a,i2,i,i1]
  GET t1a_old[a,i2]

  # the runtime has to first wait for data to be ready before executing block operation
  Tii[i1,i]        = VSpipi[a,i2,i,i1]*t1a_old[a,i2]

  # sends data back to the server
  # synchronous operation, the runtime has to wait until the transfer is completed
  PUT Fmi_a[i1,i] += Tii[i1,i]

ENDPARDO i1, i, a, i2
\end{lstlisting}

It is clear from the SIAL code fragment in Listing~\ref{alg:SIALWorkFLow} that the computing resources are wasted while
waiting for the data to be ready. To improve the wait time of the compute
operation, non-blocking MPI call \texttt{MPI\_Irecv} was exploited to prefetch
the blocks from servers over the network.

\section{Implementation of Prefetching}

\subsection{pardo Loop Implementation}
To determine which block should be prefetched, the runtime needs to predict
the next block in the loop. This depends on what type of loop is being executed.
There are two types of looping construct in SIAL: the \texttt{do} loop that iterates
over the indices one by one and the \texttt{pardo} loop that distributes the
iterations among workers. There are multiple implementations of \texttt{pardo} loop, which differ
in the distribution of indices and thus the distribution of blocks over workers:
\begin{itemize}
\item \texttt{SequentialPardoLoop}: it behaves similar to a simple do loop,
  except this loop, can loop over multiple indices.
\item \texttt{StaticTaskAllocParallelPardoLoop}: the indices for this loop are
  determined statically by distributing the block over workers in a cyclic fashion.
\item \texttt{BalancedTaskAllocParallelPardoLoop}: to support symmetric arrays,
  SIAL has \texttt{where} construct in loops which prunes the iteration based on
  some programmer-defined condition. Due to such pruning, there is a non zero
  probability that all of the iterations are assigned to one particular worker.
  This loop evaluates the \texttt{where} clauses and distributes the valid iteration
  over workers in a balanced way.
\item \texttt{FragLoopManager} and its subclasses: SIAL supports large sparse
  arrays. To loop over them efficiently SIAL has various implementations of
  fragmented pardo loops. These loops have knowledge of the internal structure of the
  sparse arrays and thus can skip over rows and columns having no useful values.
\end{itemize}
Due to so many varieties of implementation of \texttt{pardo} looping construct
and to support future implementations of indices generation schemes, it is
important to keep the mechanism of prefetching separate from indices generation.
For this, a lazy prefetching mechanism was implemented which will probe for indices
as needed, dynamically. A lazy implementation would also give freedom to vary the number
of prefetched blocks at runtime.

\subsection{Lazy Indices Probing}
Each class implementing \texttt{pardo} have a function \texttt{update} which
calculates the values of the set of indices and populates the interpreter state. This
state is used by interpreter to calculate blocks using an array and index values.

%\hline
\begin{algorithm}  {$update\_indices() \rightarrow bool$}
%\hline %
\singlespacing

\begin{algorithmic}[1]
%\caption{Processing large array in SIAL}\label{alg:euclid}
\Procedure{update\_indices}{}
\ForAll{$indices\ in\ loop$}
\State $old\_index\_val \gets
interpreter\_state.indices[index\_slot]$\Comment{get current index value}
\State $new\_val \gets old\_index\_val + 1$\Comment{Increment the index as needed}
\If{$new\_val \geq upper\_bound[index\_slot]$}
  \State $new\_val \gets lower\_seg[index\_slot]$
\EndIf
\State $interpreter\_state.indices[index\_slot] \gets new\_val$\Comment{update
  the interpreter state}
\EndFor
\If{$all\ indices\ reached\ upper\_bound$}
  \State \Return{false}
\Else
  \State \Return{true}
\EndIf
\EndProcedure
%\hline
\end{algorithmic}
\end{algorithm}

To implement lazy probing, the work done by procedure \texttt{update} is divided
into multiple procedures:

\begin{itemize}
\item \texttt{get\_next\_indices} produces set of \textit{next} indices purely
  based on indices passed as a parameter rather than getting directly from interpreter
  state. This allows us to produce series of indices independent of the state of
  the interpreter.
  \begin{algorithm}  {get\_next\_indices([index]) $\rightarrow$ [index]}
    \singlespacing

    \begin{algorithmic}[1]
      \Function{get\_next\_indices}{$current\_indices$}
      \ForAll{$index\_id\ in\ loop$}
      \State $old\_index\_val \gets current\_indices[index\_id]$\Comment{get current index value}
      \State $new\_val \gets old\_index\_val + 1$\Comment{Increment the index as needed}
      \State $new\_indices \gets new\_val$\Comment{update the index into new set of indices}
      \EndFor%
      \State \Return{$new\_indices$}\Comment{return new set independent of interpreter state}
      \EndFunction
      % \hline
    \end{algorithmic}
  \end{algorithm}

\item \texttt{peek\_indices} returns set of indices and internally takes care of
  maintaining and creating a list of indices \textit{lazily}. It calls the procedure
  \texttt{get\_next\_indices} to produce next set of indices by passing the last
  set of indices in the list as needed. It increases the length of the list by 1 if
  the set of indices requested for is the last one on the list and there are more indices
  in the loop.

  \begin{algorithm}  {peek\_indices(IndexList::iterator) $\rightarrow$ [index], IndexList::iterator}
    \singlespacing

    \begin{algorithmic}[1]
      \Function{peek\_indices}{$it$}
      \If{$IndexList.empty()$}
      \State \Return $[\ ]$
      \Else
      \State $peekedIndices \gets *it$
      \If{$next(it) == IndexList.end()$}
      \State $new\_indices \gets get\_next\_indices(*it)$
      \State $IndexList.insert\_after(it, new\_indices)$
      \EndIf
      \State \Return $peekedIndices, it$
      \EndIf
      \EndFunction
      % \hline
    \end{algorithmic}
  \end{algorithm}

\item \texttt{prefetch\_indices} remember the last set of indices returned to each
  \texttt{GET} statement and returns the next set of indices when it is called. It remembers
  that by mapping the position of indices in the list to \texttt{line numbers} of each
  \texttt{GET}. This makes varying the number of prefetched blocks for each
  \texttt{GET} possible.

  \begin{algorithm} {prefetch\_indices() $\rightarrow$ [index]}
    \singlespacing

    \begin{algorithmic}[1]
      \Function{prefetch\_indices}{}
      \State $line\_number \gets Interpreter.current\_line\_number()$

      \If{$prefetch\_map.contains(line\_number)$}
      \State $it \gets prefetch\_map[line\_number]$
      \Else
      \State $it \gets prefetch\_map.begin()$
      \EndIf
      \State $\{prefetched\_indices, it\} \gets peek\_indices(it)$
      \State $prefetch\_map[line\_number] \gets it$
      \State \Return $prefetch\_indices$
      \EndFunction
    \end{algorithmic}
  \end{algorithm}

  \begin{figure}[h] %place figure "here"
    \centering

    \tikzstyle{table}=[
    matrix of nodes,
    row sep=-\pgflinewidth,
    column sep=-\pgflinewidth,
    nodes={rectangle,draw=black,text width=15ex,align=center},
    text depth=0.25ex,
    text height=1.5ex,
    nodes in empty cells]
    \begin{tikzpicture}[list/.style={rectangle split, rectangle split parts=7,
        draw, rectangle split horizontal}, >=stealth, start chain]

      \node[list,on chain] (A) {\nodepart{one}   1 \nodepart{two}  1
                                \nodepart{three} 1 \nodepart{four} 1
                                \nodepart{five}  1 \nodepart{six}  1};
      \node[list,on chain] (B) {\nodepart{one}   1 \nodepart{two}  1
                                \nodepart{three} 1 \nodepart{four} 1
                                \nodepart{five}  1 \nodepart{six}  4};
      \node[list,on chain] (C) {\nodepart{one}   1 \nodepart{two}  1
                                \nodepart{three} 1 \nodepart{four} 1
                                \nodepart{five}  1 \nodepart{six}  7};
      \node[on chain,draw,inner sep=6pt] (D) {};
      \draw (D.north east) -- (D.south west);
      \draw (D.north west) -- (D.south east);
      \draw[*->] let \p1 = (A.seven), \p2 = (A.center) in (\x1,\y2) -- (B);
      \draw[*->] let \p1 = (B.seven), \p2 = (B.center) in (\x1,\y2) -- (C);
      \draw[*->] let \p1 = (C.seven), \p2 = (C.center) in (\x1,\y2) -- (D);

      \matrix[table,below=of A] (map)
      {
        {GET Statement}& {Set of Indices} \\
        {101}          & {}               \\
        {102}          & {}               \\
        {\vdots}       & {\vdots}         \\
      };

      \draw[*->] (map-2-2.center) to[out=0, in=270] (B.south);
      \draw[*->] (map-3-2.center) to[out=0, in=270] (C.south);
    \end{tikzpicture}
    \caption{\texttt{prefetch\_indices} saves mapping of line number to position in the list} \label{fig:prefetch_indices map}
  \end{figure}

\item \texttt{update} is now changed to simply pop the first set of indices from
  the list and update interpreter state so that other modules can find the value
  of current indices. This decreases the length of the list by 1.
  \begin{algorithm}  {update\_indices() $\rightarrow$ bool}
    \singlespacing

    \begin{algorithmic}[1]
      \Function{update\_indices}{}
      \State $current\_indices \gets peek\_indices(IndexList.begin())$
      \If{$length(current\_indices) > 0$}\Comment{is there any more iteration}
      \State $Indexlist.pop()$
      \ForAll{$index\_slot\ in\ current\_indices $}
      \State $interpreter\_state.indices[index\_slot] \gets current\_indices[index\_slot]$
      \EndFor
      \State \Return $true$
      \Else
      \State \Return $false$
      \EndIf
      \EndFunction
      % \hline
    \end{algorithmic}
  \end{algorithm}
\end{itemize}

In all, the functions \texttt{peek\_indices} and \texttt{update} can be modeled
as producer and consumer problem on a bounded buffer. Function \texttt{prefetch}
indices is free to point at any set of indices on the list.

  \begin{figure}[h] %place figure "here"
    \centering
    \begin{tikzpicture}[
      list/.style={rectangle split, rectangle split parts=7,
        draw, rectangle split horizontal},
      function/.style={rectangle, draw,align=center},
      >=stealth, start chain]

      \node[list,on chain] (A) {\nodepart{one}   1 \nodepart{two}  1
                                \nodepart{three} 1 \nodepart{four} 1
                                \nodepart{five}  1 \nodepart{six}  1};
      \node[list,on chain] (B) {\nodepart{one}   1 \nodepart{two}  1
                                \nodepart{three} 1 \nodepart{four} 1
                                \nodepart{five}  1 \nodepart{six}  4};
      \node[list,on chain] (C) {\nodepart{one}   1 \nodepart{two}  1
                                \nodepart{three} 1 \nodepart{four} 1
                                \nodepart{five}  1 \nodepart{six}  7};
      \node[on chain,draw,inner sep=6pt] (D) {};
      \draw (D.north east) -- (D.south west);
      \draw (D.north west) -- (D.south east);
      \draw[*->] let \p1 = (A.seven), \p2 = (A.center) in (\x1,\y2) -- (B);
      \draw[*->] let \p1 = (B.seven), \p2 = (B.center) in (\x1,\y2) -- (C);
      \draw[*->] let \p1 = (C.seven), \p2 = (C.center) in (\x1,\y2) -- (D);

      \node[function, below=of A]        (update)   {\texttt{update}};
      \node[function, right=of update]   (prefetch) {\texttt{prefetch\_indices}};
      \node[function, right=of prefetch] (peek)     {\texttt{peek\_indices}};

      \draw[->]                (update.north)   -- (A.south);
      \draw[densely dotted,->] (prefetch.north) -- (A.south);
      \draw[densely dotted,->] (prefetch.north) -- (B.south);
      \draw[densely dotted,->] (prefetch.north) -- (C.south);
      \draw[->]                (peek.north)     -- (C.south);
    \end{tikzpicture}
    \caption{\texttt{update} consumes, \texttt{peek\_indices} produces if needed, \texttt{prefetch\_indices} is free to move along list} \label{fig:producer-consumer}
  \end{figure}

\section{Experiments}
This section presents several experiments conducted to investigate the optimal
parameters and tradeoffs involved in the selection of parameters.

\subsection{\texttt{hit\_ratio}}\label{sec:hit_ratio}
To understand the performance of the prefetching mechanism, a new metric is introduced.
Prefetch \texttt{hit\_ratio} is defined as the ratio of the number of times the
SIA runtime did not have to block for a certain data block to be ready and the total
number of times the data block was accessed:
\[
  \texttt{hit\_ratio} = \frac{number~of~times~no~blocking~required}{total~number~of~times~data~accessed}
\]
The \texttt{hit\_ratio} represents the number of times prefetching was successful
to hide network transfer cost. In the following experiments \texttt{hit\_ratio}
will be used to measure the effectiveness of parameters in prefetching.

\subsection{Index Range}
The range of indices is the range of values the indices in the loop can take.
The range of indices are defined by the SIAL programmer at the start of the program.
The range of indices can have a high impact on prefetching. To study the relation
between index range and prefetching, \texttt{hit\_ratio} is observed by varying
the range of indices. This is presented in figure~\ref{fig:hitratio}.
\begin{figure}[h]
  \input{results/index_length/hitratio}
  \caption{Index range vs \texttt{hit\_ratio}}
  \label{fig:hitratio}
\end{figure}

Note that the runtime has to block for data only the first time it accesses a block.
Subsequent accesses do not need any blocking since the data is ready. Hence, the
\texttt{hit\_ratio} with no prefetching is non zero.

If an index spans only 1 then there is no scope for the runtime to do prefetching.
This is evident from the plot when index range is 1, the \texttt{hit\_ratio} with
prefetching is equal to with no prefetching. As the range of index is increased,
prefetching got working. This can be easily observed from exponential growth in
\texttt{hit\_ratio}. Eventually, the curve for \texttt{hit\_ratio} flattens out
after 6, since no significant improvement is achieved by increasing the index range.

It is observed that as the runtime prefetches multiple blocks,
the first request to the server takes longer as the number of index range increases. This
side effect can be explained using the preceding observation about \texttt{hit\_ratio}.
Since the increase in index range activates prefetching, network congestion
increases and the first request
to the server becomes costlier. This is presented in figure~\ref{fig:p_first_mean}.
\begin{figure}[h]
  \input{results/index_length/p_first_mean}
  \caption{Index range vs \texttt{wait\_time\_} for first iteration}
  \label{fig:p_first_mean}
\end{figure}

It can be concluded from previous observation that prefetching increases the wait time
for the first request to the server. Thus, to compensate for the high cost of the first
iteration by offsetting it in subsequent iterations, the index range
should be sufficient enough. The mean time taken per iteration is plotted against
the index range in figure~\ref{fig:p_np_mean}.

\begin{figure}[h]
  \input{results/index_length/p_np_mean}
  \caption{Index range vs \texttt{wait\_time\_} per iteration in prefetched and no prefetched loop}
  \label{fig:p_np_mean}
\end{figure}

The range of an index should be greater than five to decrease the \texttt{wait\_time\_}
by a factor of two. The mean \texttt{wait\_time} per iteration with prefetching can
reduce up to 3 times, as compared to with no prefetching, if the index
range is greater than nine.

\subsection{Block Size}
Since the time to transfer block over the network is proportional to the size of the block, the
block size affects the first request made during prefetching. Along with the first
request, multiple requests for prefetching subsequent blocks are made. This makes
the \texttt{wait\_time\_} for first call sensitive to block size. This is evident
from the graph plotting Block Size against mean \texttt{wait\_time} for the first
iteration in figure~\ref{fig:first_wait_time}.
\begin{figure}[h]
  \input{results/block_size/first_wait_time}
  \caption{Block size vs \texttt{wait\_time\_} for first iteration}
  \label{fig:first_wait_time}
\end{figure}

For the first iteration, \texttt{wait\_time\_} in the case of loops with prefetching
grows faster than loops without prefetching. At size of block
greater than 500 elements, \texttt{wait\_time\_} for the first iteration with prefetching
is almost twice the corresponding \texttt{wait\_time\_} without prefetching.

It seems that subsequent iterations are not affected by the block size in the loops
with prefetching, since the runtime need not block for subsequent blocks. This results in reduction
of overall mean \texttt{wait\_time\_}. This trend is presented in figure~\ref{fig:block_size_avg_wait_time}.

\begin{figure}[h]
  \input{results/block_size/avg_wait_time}
  \caption{Block size vs mean \texttt{wait\_time\_} for prefetched and no prefetch loop}
  \label{fig:block_size_avg_wait_time}
\end{figure}

The mean \texttt{wait\_time\_} grows much slower for loops with prefetching as compared
to loops without prefetching.

All of these trends of block size against first and mean \texttt{wait\_time\_} for
loops with and without prefetching are summarized in figure~\ref{fig:block_size_avg_all}
\begin{figure}[h]
  \input{results/block_size/avg_all}
  \caption{Block size vs mean \texttt{wait\_time\_} for prefetched and no prefetched loop}
  \label{fig:block_size_avg_all}
\end{figure}

Although the \texttt{wait\_time\_} for the first iteration grows at the highest rate,
prefetching compensates for it and keeps the mean \texttt{wait\_time\_} lower than
loops without prefetching.

\subsection{Number of Blocks to Prefetch}
As stated in previous sections, prefetching affects the first request to the server
and the size of the block also affects the first request. To observe the effect of
the number of blocks to prefetch on the initial request, the number of blocks were
varied and plotted against mean \texttt{wait\_time\_} for the first request. This is
presented in figure~\ref{fig:look_ahead_first_wait_time}.
\begin{figure}[h]
  \input{results/look_ahead/first_wait_time}
  \caption{Number of block prefetched vs \texttt{wait\_time\_} for the first request}
  \label{fig:look_ahead_first_wait_time}
\end{figure}

It is clear from the plot that the mean \texttt{wait\_time\_} for the first request
grows linearly with the number of blocks to prefetch. Thus the number of blocks
to prefetch cannot be set at very high number unless the index range is
known to be large.

To determine the effect of the number of blocks to prefetch to prefetching, the number
of blocks to prefetch was varied and is plotted against the mean \texttt{wait\_time\_}.
This plot is presented in figure~\ref{fig:look_ahead_avg_wait_time}.
\begin{figure}[h]
  \input{results/look_ahead/avg_wait_time}
  \caption{Number of block prefetched vs mean \texttt{wait\_time\_}}
  \label{fig:look_ahead_avg_wait_time}
\end{figure}

For the case when the number of blocks prefetched is zero, which is in the case of no
prefetching, the mean \texttt{wait\_time\_} is highest. It drops sharply as the
number of blocks to prefetch increases. It grows again, since with an increase
in the number of blocks to prefetch, the first request to the server gets expensive.

As the number of blocks to prefetched is increased, more blocks are available for
runtime without blocking. This is presented in figure~\ref{fig:look_ahead_hit_ratio}
by plotting number of blocks to prefetch against \texttt{hit\_ratio}.

\begin{figure}[h]
  \input{results/look_ahead/hit_ratio}
  \caption{Number of block prefetched vs the hit ratio for the first request}
  \label{fig:look_ahead_hit_ratio}
\end{figure}

The hit ratio saturates after hitting a critical amount. After that increase, the number
of blocks to prefetch has no effect. This explains the rise in mean
\texttt{wait\_time\_} as \texttt{wait\_time\_} for the first request grows and the
number of blocks available without blocking stays constant.

\subsection{Overall Improvement}
\subsubsection{$C_{12}H_{10}(BP)$ molecule}\label{sec:bp_molecule}
To study the effect of prefetching on real-world application, a job from ACES4 on
$C_{12}H_{10}$ molecule in a CC-PVDZ basis was executed. The total number of workers were 3 and number
of servers were varied to study its effect on prefetching. This is plotted in
figure~\ref{fig:prefetch_real}.

\begin{figure}[h]
  \input{results/prefetch_real/servers}
  \caption{Effects of varying number of servers on $C_{12}H_{10}$ molecule.}
  \label{fig:prefetch_real}
\end{figure}

With the increase in the number of servers, the overall runtime for programs with
prefetching decreases. This is clear from figure~\ref{fig:prefetch_real}, as the difference
in the runtime of programs with and without prefetching increase. It is worth noticing
that the only couple of programs show a significant increase in difference while one
of the programs shows no difference in prefetching and no prefetching. To study
why this is the case, the total \texttt{wait\_time} for the programs is plotted in
figure~\ref{fig:prefetch_real_barrier}.

In the figure~\ref{fig:prefetch_real_barrier}, along with \texttt{wait\_time},
\texttt{barrier wait\_time} is also plotted for all the programs involved. The
\texttt{wait\_time} of program \texttt{scf\_rhf} is mostly due to \texttt{barrier wait\_time}
and rest two have considerably less \texttt{barrier wait\_time} as compared to
overall \texttt{wait\_time}. This explains why only two of the 3 programs showed
significant improvement by prefetching since prefetching helps in reducing \texttt{wait\_time}
for the block to be transferred over the network. The program \texttt{sch\_rhf} did not
have much \texttt{wait\_time} for network transfer. The overall improvement in
\texttt{wait\_time} for CCSD calculation is about 27\%.

\begin{figure}[h]
  \input{results/prefetch_real/barrier_wait_time}
  \tikz{\path[draw,fill={rgb:red,97;green,156;blue,45}] (0,0) rectangle (1cm,0.5cm);} w/o Prefetch Total \texttt{wait\_time}.
  \tikz{\path[draw,fill={rgb:red,254;green,56;blue,34}] (0,0) rectangle (1cm,0.5cm);} w/o Prefetch Barrier \texttt{wait\_time}
  \\
  \tikz{\path[draw,fill={rgb:red,30;green,185;blue,3}] (0,0) rectangle (1cm,0.5cm);} w/~~Prefetch Total \texttt{wait\_time}.
  \tikz{\path[draw,fill={rgb:red,195;green,3;blue,3}] (0,0) rectangle (1cm,0.5cm);}  w/~~Prefetch Barrier \texttt{wait\_time}
  \caption{Overall \texttt{wait\_time} improvement in $C_{12}H_{10}$ molecule in a CC-PVDZ basis}
  \label{fig:prefetch_real_barrier}
\end{figure}

\subsubsection{$C_{20}$ molecule}
Another, much bigger, calculation was executed to test the effect of prefetching
in presence of higher number of participating nodes. There were two runs of the
calculation, first with 512 CPUs and second with 32 CPUs. In both of cases, the
number of workers and servers were kept equal to maximize the effect of prefetching.

Results of executing CCSD calculation on $C_{20}$ molecule in a AUG-CC-PVDZ basis is
plotted in figure~\ref{fig:prefetch_real_256}.

\begin{figure}[h]
  \input{results/prefetch_real/rccsd_rhf256}
  \\
  \tikz{\path[draw,fill={rgb:red,97;green,156;blue,45}] (0,0) rectangle (1cm,0.5cm);} w/o Prefetch Total \texttt{wait\_time}.
  \tikz{\path[draw,fill={rgb:red,254;green,56;blue,34}] (0,0) rectangle (1cm,0.5cm);} w/o Prefetch Barrier \texttt{wait\_time}
  \\
  \tikz{\path[draw,fill={rgb:red,30;green,185;blue,3}] (0,0) rectangle (1cm,0.5cm);} w/~~Prefetch Total \texttt{wait\_time}.
  \tikz{\path[draw,fill={rgb:red,195;green,3;blue,3}] (0,0) rectangle (1cm,0.5cm);}  w/~~Prefetch Barrier \texttt{wait\_time}
  \caption{Overall \texttt{wait\_time} improvement in CCSD calculation for $C_{20}$ molecule in a AUG-CC-PVDZ basis with 512 CPUs}
  \label{fig:prefetch_real_256}
\end{figure}

The overall improvement in the case of 512 CPUs is not significant. It is 2\%. This
can be attributed to the number of participating workers in the calculation. Since
the number of workers was very high, the index range got distributed over a large
number of workers and each worker got insufficient range for prefetching. To confirm
this is the case, another experiment with same molecule and same basis function
was conducted, with much lesser number of workers. The result for this calculation
is presented in figure~\ref{fig:prefetch_real_32}.

\begin{figure}[h]
  \input{results/prefetch_real/rccsd_rhf32}
  \\
  \tikz{\path[draw,fill={rgb:red,97;green,156;blue,45}] (0,0) rectangle (1cm,0.5cm);} w/o Prefetch Total \texttt{wait\_time}.
  \tikz{\path[draw,fill={rgb:red,254;green,56;blue,34}] (0,0) rectangle (1cm,0.5cm);} w/o Prefetch Barrier \texttt{wait\_time}
  \\
  \tikz{\path[draw,fill={rgb:red,30;green,185;blue,3}] (0,0) rectangle (1cm,0.5cm);} w/~~Prefetch Total \texttt{wait\_time}.
  \tikz{\path[draw,fill={rgb:red,195;green,3;blue,3}] (0,0) rectangle (1cm,0.5cm);}  w/~~Prefetch Barrier \texttt{wait\_time}
  \caption{Overall \texttt{wait\_time} improvement in CCSD calculation for $C_{20}$ molecule in a AUG-CC-PVDZ basis with 32 CPUs}
  \label{fig:prefetch_real_32}
\end{figure}

The overall improvement, in \texttt{wait\_time} in the case of 32 CPUs, is 49\%.
The range of index was divided over much smaller number of workers and hence
each one of the 16 workers had enough index range to prefetch data blocks.