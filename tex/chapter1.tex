\chapter{INTRODUCTION TO THESIS} \label{intro}

The Super Instruction Architecture (SIA) is a parallel programming environment
originally designed for problems in computational chemistry involving complicated
expressions defined in terms of tensors. Tensors are represented by
multidimensional arrays which are typically very large. The SIA consists of a
domain specific programming language, Super Instruction Assembly Language
(SIAL), and its runtime system, Super Instruction Processor. An important
feature of SIAL is that algorithms are expressed in terms of blocks or
multidimensional arrays rather than individual floating point numbers.

\section{Introduction to Issues with Working with GPUs}

In ACESIII, the previous version of ACES, programmer had to deal with managing
memory for GPU and marking regions well suited for execution on GPU. In ACES4,
managing memory is done automatically by the runtime. This is implemented using
version numbers for block data.

But still there is need for manual intervention to determine which
portion of the SIAL code is well suited for execution on GPU. This is not a
trivial decision to make since performance of GPU v/s CPU depends on various factors such as
at runtime. Larger blocks need more time to transfer between GPU and CPU, but at
size of the block and surroudning instructions in the program. Judging
based on size of block is itself difficult because of the type of GPU available
same time the difference in time taken to operate on blocks by GPU and CPU grows
exponentially with the size of the block. And lastly, same programs are used to
calculate different results by supplying different data files. A portion of code
which executed x times faster on GPU may in fact execute slower than CPU on
smaller size of block.

\section{Organization of the Thesis}
Chapter 2 presents related literature regarding efforts made to exploit GPU in
SIA. Then, an introduction to background of this thesis, including architecture
and implementation of SIA and ACES4, is presented in chapter 3. Chapter 4
describes the problem and solution of optimizing use of GPU to execute SIAL
code; the design and implementation of eager-pushing of blocks and dynamic
determination of suitable code block for execution on GPU is presented in
chapter 5. Chapter 6 states the results of the benchmarks and experiments. And
finally, chapter 7 presents conclusion of this thesis and future works.