\chapter{RELATED WORK} \label{lit}

We discuss work done by others in using GPU for scientific calculations using
several methods such as directive based GPU programming models, using Cuda in
higher level programming language and use of GPU in previous versions of ACES.

\section{Directive Based GPU programming}
There are several attempts in area of directive based programming models. These
attempts include OpenACC \cite{openacc}, OpenMP for accelerator
\cite{openmpforaccelerators}, and less updated attempts including OpenMPC
\cite{openmpc} and hiCuda \cite{hicuda}. The approach taken by these models
is to make a implementation GPU ready by placing directives which made
execution of loops parallel. In contrast, SIA exploits coarse grained
concurrency using super instructions and super numbers or blocks by executing
super instruction on GPU. These super instructions can in turn make use of above
models to make an existing CPU implementation GPU ready by inserting suitable
directives or have a completely rewritten low level Cuda implementation. Thus
these models works \textit{with} the previously mentioned models rather than in
competition with.

\subsection{GPU in Computational Chemistry}

There are implementations of coupled cluster methods in computational chemistry
on GPUs reported \cite{bhaskar2013}\cite{deprince2011}\cite{maw2011}. In these
implementations, the a specific algorithm was selected and then implemented on a
GPU in a highly optimized form. These implementations have hardware specific
optimizations, and not generic enough to be effective on new hardware.

One such implementation of contraction operators on a GPU by Ma et al. generates
CUDA code to directly implement the contractions and optimizes for the
particular order of indices in the contractions. This, as compared to using
permutations and then two dimensional matrix-matrix multiply as implemented in
SIA, should achieve better performance for a contraction.

\section{GPU Programming in Other High Level Programming Languages}

Several attempts for support for GPU in general purpose high level programming
languages have been made. A few examples include PyCUDA \cite{pycuda2011}, jCUDA
\cite{jcuda2009} and Chapel \cite{chapelgpu}. These toolkits gives an interface
to GPU from high level programming languages such as Python and Java but
programmer still needs to deal with low level GPU book keeping such as memory
management and defining kernels. Whereas in SIA, once the superinstructions are
defined, runtime can jump between GPU and CPU implementation transparently to
the domain programmer, the runtime takes care of memory management and GPU
implementations of common operations on blocks are build into the runtime.

\section{GPU in Previous Versions of ACES}

There has been work done previously to exploit GPU in Super Instruction
Architecture by \cite{jindal2016gpusial} in ACESIII. This implementation
required programmer to mark a block of code to be executed on GPU using
directives \textbf{gpu\_on} and \textbf{gpu\_off} and manually manage memory
transfer to and from GPU. Executing operations on small blocks on GPU can
actually be slower than executing on CPU due to time lost in memory transfer and
insignificant gain in speed. Thus it is better to execute blocks with 2 indices
on CPU and block with 4 indices and up on GPU. SIAL programmer can take this
call of which block is large enough so that benefit of executing it on GPU
becomes significant.