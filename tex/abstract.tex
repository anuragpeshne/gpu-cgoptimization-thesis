% Write in only the text of your abstract, all the extra heading jargon is automatically taken care of
\begin{abstract}
  The Super Instruction Architecture (SIA) is a parallel programming environment
  engineered to work on large blocks of floating point numbers. Looping constructs
  \texttt{do} and \texttt{pardo} are one of the most important constructs in Super
  Instruction Assembly Language (SIAL) since they allow the programmer to work
  with an individual block. To improve the runtime performance of the looping constructs,
  two techniques are used: improving GPU utilization and prefetching blocks to
  hide network latency.\\

  In the previous versions, the domain programmer was needed to manually transfer blocks between main memory
  and GPU memory as well as to mark parts of SIAL code to be executed on GPU. Copying data
  between GPU memory and main memory is costly and have a high impact on overall
  performance of GPU execution. Automatic memory management is provided
  and various techniques are implemented to reduce and in some cases eliminate
  GPU memory and main memory transfer cost.\\

  There is a common code pattern in SIAL to request a block of data followed by
  computation on that block. This pattern makes inefficient use of network
  and compute resources, since one remains underutilized while the other is being used.
  Deterministic request for blocks has been exploited
  to implement prefetching to optimally utilize the network resources during the time
  computational resources have blocked the control flow.\\

  Various experiments have been conducted to evaluate the effect of change of various
  parameters on GPU utilization, prefetching, \texttt{wait\_time} and overall
  computation time.
\end{abstract}
