\chapter{INTRODUCTION}\label{intro}
The Super Instruction Architecture (SIA)\cite{Sanders:2010:BLR:1884643.1884677} is a parallel programming environment
originally designed for problems in computational chemistry involving complicated
expressions defined in terms of tensors. Tensors are represented by
multidimensional arrays which tend to be very large in a typical computation chemistry
calculation. SIA consists of a
domain-specific programming language, Super Instruction Assembly Language
(SIAL), and its runtime system, Super Instruction Processor (SIP). An important
feature of SIAL is that algorithms are expressed in terms of blocks or
multidimensional arrays rather than individual floating point numbers. This thesis
presents two ideas to enhance the looping constructs in SIAL.

\section{Issues of Working with GPUs}
Advanced Concepts in Electronic Structure (ACES)\cite{doi:10.1002/qua.560440876, doi:10.1002/wcms.77} is an implementation of SIA for
chemical computation.
In ACESIII\cite{Jindal2016}, the previous version of ACES, the programmer had to deal with explicitly managing
the memory transfer between GPU and main memory, and marking regions that are well suited for execution
on a GPU. In ACES4, memory transfer is managed automatically by the runtime.
This is implemented by keeping a version number associated with each device memory
and tracking changes to blocks by every compute device, by incrementing this version
number.

Nonetheless, there is a need for the SIAL programmer to decide which portion of the SIAL code
is suited for execution on GPU. Transferring data between GPU and main memory is
expensive\cite{datatransferoptimization}, and can dominate the overall time spent in computation. Therefore, marking
correct block of code, that is suitable for execution on GPU, is not
a trivial decision. Time taken for transfering data depends on various factors such as
the size of the blocks involved and operations involved in the computation.
Larger blocks need more time to transfer between GPU and main memory, but at the
same time, the speed improvement obtained by execution on GPU grows
with the size of the block. Lastly, the same SIAL programs are used to
calculate different results by supplying different data files. It is possible that,
for the same program, the overall time required for execution on GPU be greater
than CPU for some size of data and less than CPU for another size of data.
This may happen if the speed gains by executing on GPU cannot compensate for the
transfer time.

\section{Issues in Transfer of Data from Servers}
In SIAL, looping constructs are the only way to work with individual blocks.
The typical workflow of working with large arrays in SIAL includes requesting a
block of the larger array from a server,
processing the block, computing resulting block and sending it back to a server. Though
the operation of requesting a block is nonblocking, subsequent operations
which operate on the block, will wait until Message Passing Interface (MPI)
transfer is completed. To reduce the overall cost of network
transfer, prefetching of blocks has been implemented, which does the transfer concurrently
with the block processing. By prefetching the blocks that are anticipated in
the next iteration of the loop, the wait~time for a block to be ready
can be reduced and in some cases completely eliminated.

\section{Organization of the Thesis}
Chapter 2 presents literature study of the previous work done in use of GPU for general purpose
computing, efficient GPU memory transfer, and the technique of prefetching to hide
latency in accessing the data. Then a background of this thesis, including architecture
and implementation of SIA and ACES4, is presented in chapter~3.
Chapter 4 describes the technique of block prefetching implemented in SIA. Chapter 5
presents the optimizations done for efficient GPU memory transfer. Chapter 6 states the
results of the benchmarks and experiments. And
finally, chapter 7 concludes this thesis by presenting the conclusion and scope
of future works.
