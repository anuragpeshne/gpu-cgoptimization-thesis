\chapter{EXPERIMENTS AND RESULTS}\label{results}

This thesis present a few of the possible performance optimizations on looping
constructs for SIA. These optimizations primarily aimed to reduce the network
operation cost and computing cost. This section describes series of experiments
conducted by varying input parameters such as block size as well as the optimization
parameters such as number of blocks to prefetch and its effects on the performance
of the system.

\section{Environment}
These experiments were carried on HiperGator Computer at UF. Table~\ref{tab:hpg2spec}
describes the specification of HiperGator 2. Table~\ref{tab:hpgcomputespecs}
explains the specifications of HiperGator 2 \textbf{compute} nodes and
table~\ref{tab:hpggpuspecs} explains the specifications of HiperGator 2 \textbf{GPU}
nodes. The table~\ref{tab:hpgconnectspecs} describes the specification for the
node interconnect in HiperGator 2.

\begin{table}[h]
  \centering
  \begin{tabular}{l | c}
    \hline
    Name        & Specifications  \\
    \hline
    Total Cores & 30,000          \\
    Memory      & 120 Terabytes   \\
    Storage     & 1 Petabytes     \\
    Max Speed   & 1,100 Teraflops \\
    \hline
  \end{tabular}
  \caption{HiperGator 2 Spec Sheet}
  \label{tab:hpg2spec}
\end{table}

\begin{table}[h]
  \centering
  \begin{tabular}{l | c}
    \hline
    Name                       & Specification     \\
    \hline
    Manufacturer               & Dell Technologies \\
    Processor                  & Intel E5-2698v3   \\
    Base Processor Frequency   & 2.3 GHz           \\
    Sockets                    & 2                 \\
    Cores per socket           & 16                \\
    Thread(s) per core         & 1                 \\
    Memory per node            & 128 Gigabytes     \\
    Memory Frequency           & 2133 MHz DDR4     \\
    \hline
  \end{tabular}
  \caption{HiperGator 2 \textbf{Compute} Node}
  \label{tab:hpgcomputespecs}
\end{table}

\begin{table}[h]
  \centering
  \begin{tabular}{l | c}
    \hline
    Name                       & Specification     \\
    \hline
    Manufacturer               & Dell Technologies \\
    Processor                  & Intel E5-2683     \\
    Base Processor Frequency   & 2.0 GHz           \\
    Sockets                    & 2                 \\
    Cores per socket           & 14                \\
    Thread(s) per core         & 1                 \\
    GPU                        & Tesla K80         \\
    Memory per node            & 128 Gigabytes     \\
    Memory Frequency           & 2133 MHz DDR4     \\
    \hline
  \end{tabular}
  \caption{HiperGator 2 \textbf{GPU} Node}
  \label{tab:hpggpuspecs}
\end{table}

\begin{table}[h]
  \centering
  \begin{tabular}{l | c}
    \hline
    Name             & Specification                                 \\
    \hline
    Node Connection  & Mellanox 56Gbit/s FDR InfiniBand interconnect \\
    Core Switches    & 100 Gbit/s EDR InfiniBand standard            \\
    \hline
  \end{tabular}
  \caption{HiperGator 2 Node interconnect specification}
  \label{tab:hpgconnectspecs}
\end{table}

\section{Prefetching}
This section presents several experiments conducted to investigate the optimal
parameters and tradeoffs involved in the selection of the parameters.

\subsection{\texttt{hit\_ratio}}
To understand the performance of the prefetching mechanism a new metric is introduced.
Prefetch \texttt{hit\_ratio} is defined as the ratio of the number of times the
SIA runtime did not have to block for a certain data block to be ready and total
number of times the data block is accessed:
\[
  \texttt{hit\_ratio} = \frac{number~of~times~no~blocking~required}{total~number~of~times~data~accessed}
\]
The \texttt{hit\_ratio} represents the number of times prefetching was successful
to hide network transfer cost. In the following experiments \texttt{hit\_ratio}
will be used where appropriate to measure effectiveness of parameters in prefetching.

\subsection{Index Length}
The length of indices is the length of the range of indices involved in the loop.
The length of indices can have high impact on prefetching. To study this relation
between index length and prefetching, \texttt{hit\_ratio} is observed by varying
the range of indices. This is presented in figure~\ref{fig:hitratio}.
\begin{figure}[h]
  \input{results/index_length/hitratio}
  \caption{Index Range Length v/s \texttt{hit\_ratio}}
  \label{fig:hitratio}
\end{figure}

Note that the runtime has to block for data only first time it accesses a block.
Subsequent accesses do not need any blocking since the data is ready. Hence the
\texttt{hit\_ratio} with no prefetching is non zero.

If a index spans only 1 then there is no scope for the runtime to do prefetching.
This is evident from the plot when index length is 1, the \texttt{hit\_ratio} with
prefetching is equal to with no prefetching. As range of index length increased,
prefetching gets working. This can be easily observed from exponential growth in
\texttt{hit\_ratio}. And eventually the curve for \texttt{hit\_ratio} flattens out
after 6 since no significant improvement is achieved by increasing the index range
length.

It is observed that as the runtime requests for multiple blocks for prefetching,
the first request to server takes longer as number of index range increases. This
side effect can be explained using the preceding observation about \texttt{hit\_ratio}.
Since the increase in index range length activates prefetching the first request
to server becomes costlier. This is presented in figure~\ref{fig:p_first_mean}.
\begin{figure}[h]
  \input{results/index_length/p_first_mean}
  \caption{Index Range v/s \texttt{wait\_time\_} per iteration}
  \label{fig:p_first_mean}
\end{figure}

It can be concluded from previous observation that prefetching increases the time
for the first request to server. Thus to compensate for the high cost of first
iteration by offsetting it in subsequent iterations the length of index range to
should be sufficient enough. The mean time taken per iteration is plotted against
the length of index range in figure~\ref{fig:p_np_mean}.

\begin{figure}[h]
  \input{results/index_length/p_np_mean}
  \caption{Index Range v/s \texttt{wait\_time\_} per iteration in Prefetched and no Prefetched Loop}
\end{figure}

The length of index should be around 5 to decrease the \texttt{wait\_time\_}
by factor of 2.

\subsection{Block Size}
Block size affects the first request made during prefetching. Since along with this
request, multiple requests for prefetching are made. This evident from Block Size
v/s mean \texttt{wait\_time} for first iteration.
\begin{figure}[h]
  \input{results/block_size/first_wait_time}
  \caption{Block Size v/s \texttt{wait\_time\_} for first iteration}
\end{figure}

But once the request for blocks are made, subsequent iterations are affected less
by the block size as compared to loops in which prefetching is not done. This results
in overall reduction in mean \texttt{wait\_time\_}.
\begin{figure}[h]
  \input{results/block_size/avg_wait_time}
  \caption{Block Size v/s Mean \texttt{wait\_time\_} for Prefetched and No Prefetch Loop}
\end{figure}

This is can viewed in better way by looking at both mean \texttt{wait\_time\_} and
mean \texttt{wait\_time\_} for first iteration together.
\begin{figure}[h]
  \input{results/block_size/avg_all}
  \caption{Block Size v/s Mean \texttt{wait\_time\_} for Prefetched and No Prefetch Loop}
\end{figure}

\subsection{Number of Blocks to Prefetch}
The number of blocks to prefetch affects the initial request made.
\begin{figure}[h]
  \input{results/look_ahead/first_wait_time}
  \caption{Number of Block Prefetched v/s \texttt{wait\_time\_} for first request}
\end{figure}

To determine optimal number of block to prefetch we need to consider how much
number of block prefetching affect mean \texttt{wait\_time\_}.
\begin{figure}[h]
  \input{results/look_ahead/avg_wait_time}
  \caption{Number of Block Prefetched v/s mean \texttt{wait\_time\_}}
\end{figure}

Hit ratio saturates after hitting a critical amount. There is no much use after
that to increase number of blocks to prefetch.
\begin{figure}[h]
  \input{results/look_ahead/hit_ratio}
  \caption{Number of Block Prefetched v/s Hit Ratio for first request}
\end{figure}

\section{GPU}
\subsection{Memory Pinning}
\subsubsection{Copy Speed}
\begin{figure}[h]
  \input{results/mempin/block_copy/pin_vs_nopin}
  \caption{Time taken to transfer block to GPU for \textit{pinned} and \textit{non pinned} blocks}
\end{figure}
\subsection{Optimized Transfer}
\begin{figure}[h]
  \input{results/optimized_block_transfer/rccsd_rhf}
  \caption{Optimized v/s Unoptimized Block Tranfers for \texttt{rccsd\_rhf.sialx}}
\end{figure}

\subsection{Memory Pinning Overhead}
\subsubsection{\texttt{alloc}}
\begin{figure}[h]
  \input{results/mempin/overhead/alloc}
  \caption{Pinned and non Pinned memory allocation}
\end{figure}

\subsubsection{\texttt{free}}
\begin{figure}[h]
  \input{results/mempin/overhead/free}
  \caption{Pinned and non Pinned memory de-allocation}
\end{figure}