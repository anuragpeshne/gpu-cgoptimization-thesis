\chapter{INTRODUCTION}\label{intro}
Advanced Concepts in Electronic Structure (ACES)
is a series of computational chemistry packages used to calculate electronic energies,
emphasizing coupled cluster methods. ACESII~\cite{doi:10.1002/qua.560440876} is a serial program, ACESIII~\cite{doi:10.1002/wcms.77} and ACES4 are
both parallel programs built using SIA concepts. The Super Instruction Architecture (SIA)\cite{Sanders:2010:BLR:1884643.1884677} is a parallel programming environment
originally designed for problems in computational chemistry involving complicated
expressions defined in terms of tensors. Tensors are represented by
multidimensional arrays which tend to be very large in a typical computation chemistry
calculation. The SIA consists of a
domain-specific programming language, Super Instruction Assembly Language
(SIAL), and its runtime system.
A novel feature of the programming language is that SIAL provides intrinsic support
for partitioning arrays into blocks and the domain programmers express
algorithms in terms of operations on blocks rather than individual floating point
numbers. Two enhancements for ACES4, improvement in the GPU support and hiding of the network
latency using data prefetching, are presented in this work.\\


\section{Issues of Working with GPUs}
In ACESIII\cite{Jindal2016} the programmer had to deal with explicitly managing
the memory transfer between GPU and main memory, and marking regions that are well suited for execution
on a GPU. In ACES4, memory transfer is managed automatically by the runtime.
This is implemented by keeping a version number associated with each device memory
and tracking changes to blocks by every compute device.

Nonetheless, there is a need for the SIAL programmer to decide which portion of the SIAL code
is suited for execution on GPUs. Transferring data between GPU and main memory is
expensive\cite{datatransferoptimization}, and can dominate the overall time spent in computation. Therefore, marking
correct block of code, that is suitable for execution on GPU, is not
a trivial decision. Time taken for transfering data depends on various factors such as
the size of the blocks involved and operations involved in the computation.
Larger blocks need more time to transfer between GPU and main memory, but at the
same time, the speed improvement obtained by execution on GPU grows
with the size of the block. Lastly, the same SIAL programs are used to
calculate different results by supplying different input parameters. It is possible that,
for the same program, the overall time required for execution utilizing GPUs be greater
than CPU only for some size of data and less than CPU for another size of data.
This may happen if the speed gains by executing on GPUs cannot compensate for the
transfer time.

\section{Issues in Transfer of Data from Servers}
In SIAL, looping constructs are the only way to work with individual blocks.
The typical workflow of working with large arrays in SIAL includes requesting a
block of a large array from a server,
processing the block, computing resulting block and sending it back to a server. Though
the operation of requesting a block is nonblocking, subsequent operations
which operate on the block will wait until Message Passing Interface (MPI)
transfer is completed. To reduce the overall cost of network
transfer, prefetching of blocks, which does the transfer concurrently
with the block processing, has been implemented. By prefetching the blocks that are anticipated in
the next iteration of the loop, the wait~time for a block to be ready
can be reduced and in some cases completely eliminated.

\section{Organization of the Thesis}
Chapter~2 presents more details about SIA and ACES. Chapter 3 presents previous work done in use of GPUs for general purpose
computing, efficient GPU memory transfer, and the technique of prefetching to hide
latency in accessing the data.
Chapter 4 describes the technique of block prefetching implemented in SIA. Chapter 5
presents the optimizations done for efficient GPU memory transfer. And
finally, chapter 6 concludes this thesis by presenting the conclusion and suggestion
of future work.
